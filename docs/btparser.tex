% pdflatex canal.tex
\documentclass[a4paper]{book}
\usepackage[utf8]{inputenc}
\usepackage[english]{babel}
\usepackage{a4wide}
\usepackage{makeidx}
\usepackage{hyperref}
\usepackage{graphicx}
\usepackage{import}
\usepackage{amsmath}
\usepackage{amssymb}
\usepackage{txfonts}
\usepackage{float}
\usepackage{doxygen/doxygen}

\fancyfoot[RE]{\fancyplain{}{}}
\fancyfoot[LO]{\fancyplain{}{}}

\setcounter{tocdepth}{2}

\makeindex

\title{Btparser\\
\vskip 1em
\large{A program failure analysis library}}
\author{Karel Klíč}

\begin{document}
\maketitle

\tableofcontents

\cleardoublepage

\chapter{Overview}

Failures of computer programs are omnipresent in the information
technology industry: they occur during software development, software
testing, and also in production.  Failures occur in programs from all
levels of the system stack.  The program environment differ
substantially between kernel space, user space programs written in C
or C++, Python scripts, and Java applications, but the general
structure of failures is surprisingly similar between the mentioned
environments due to imperative nature of the languages and common
concepts such as procedures, objects, exceptions.

Btparser is a collection of low-level algorithms for program failure
processing, analysis, and reporting supporting kernel space, user
space, Python, and Java programs.  Considering failure processing, it
allows to parse failure description from various sources such as
GDB-created stack traces, Python stack traces with a description of
uncaught exception, and kernel oops message.  Infromation can also be
extracted from the core dumps of unexpectedly terminated user space
processes and from the machine executable code of binaries.
Considering failure analysis, the stack traces of failed processes can
be normalized, trimmed, and compared.  Clusters of similar stack
traces can be calculated.  In multi-threaded stack traces, the threads
that caused the failure can be discovered.  Considering failure
reporting, the library can generate a failure report in a
well-specified format, and the report can be sent to a remote machine.

Due to the low-level nature of the library and implementors' use
cases, most of its functionality is currently limited to Linux-based
operating systems using ELF binaries.  The library can be extended to
support Microsoft Windows and OS X platforms without changing its
design, but dedicated engineering effort would be required to
accomplish that.

\part{Concepts}

\chapter{Stack Trace Normalization}

\chapter{Stack Trace Clustering}

\chapter{Core Dump Failure Analysis}

\chapter{Wishlist}
Security Impact.

ABI compatibility check.

Collecting environment data.

\part{Implementation}

\chapter{Overview}
Btparser is implemented in the C language as defined in the C99
standard (ISO/IEC 9899:1999).  It uses the C standard library and some
additional libraries.  No additional library is mandatory, though.
When a library is not found by the build configuration script, the
features requiring that library become unavailable.  This approach
improves both usability and portability of the library.

\import{doxygen/}{refman}

\chapter{Known Bugs}
Empty.

\chapter{Wishlist}
Stack trace for kerneloopses, Python, and Java.

\clearpage
\addcontentsline{toc}{chapter}{Index}
\printindex

\end{document}
